\chapter{Software}

Die Software ist in 2 Abschnitte Aufgeteilt, einmal in den Webserver mit Webinterface und die Datenverarbeitung mit \gls{dac} Steuerung.

\section{Webserver mit Webinterface}

Der Webserver ist ein Asynchroner TCP Webserver???????????????????????????\\

Das Webinterface ist mit HTML und javascript geschrieben.
Dies ermöglicht ein einfaches aber auch modernes Designe zu erzeugen und dieses auf einem Mikrocontroller laufen zulassen.
Das Webinterface bietet die Möglichkeiten .txt Dateien auf den Mikrocontroller hochzuladen und abzuspeichern.
Diese werden dann in einer Liste angezeigt und stellen darin auch die "Abspiel" Reihenfolge dar.
Diese Reihenfolge kann mittels Verschieben $\left(Drag \ and \ Drop \right)$ geändert werden.

Um die Einzelen Dateien den Ausgangskanälen des \gls{dac} zuzuordnen ist es möglich am rechten Ende Des Dateifeldes den Kanal in einem Menü auszuwählen.
Wenn nun auf der linken Seite des Dateifeldes die Datei noch ausgewählt wird kann man die Verarbeitung der Dateien starten.
Hierfür muss der unter Button "Verarbeitung Anstoßen" betätigt werden.
Dieser führt eine Funktion aus, die die Reihenfolge der ausgewählten Dateien einließt und deren Inhalt auf das nötige Format überprüft.
Dabei werden alles Zahlen aus den Dateien heraus genommen und in der Reihenfolge innerhalb der Date und dann in der Dateireihenfolge in das dem Kanal zugehörigem Array geschrieben.

Wenn dies nun Abgeschlossen ist, kann der Button "Abspielen" betätigt werden.
Daraufhin werden die Zahlen in den Arrays an den \gls{dac} weitergegeben.
Hierfür wird jede Zahl in den passenden Wert für den Wertebereich des \gls{dac} umgewandelt. 
Dieser hat mit 12 Bit einen Wandlungsbereich von 0 bis 4095.
Da der gls{dac} Bipolar betrieben wird, liegen nun die negativen Zahlen im Bereich von 0 bis 2047 und die positiven Zahlen von 2048 bis 4095.
Wenn nun die Zahl umgewandelt wurde wird diese nun in eine 8 Bit umgewandelt und jeder einzelne Bit wird nun durch einen GPIO-Port repräsentiert.
Dadurch lässt sich mit nur einem einzelen Takt der Zustand der DAC-Eingänge und damit die komplette Zahl einlesen und übergeben.
Nach einlesen der Zahl im \gls{dac} wird diese nun Umgewandelt und als analoger Wert ausgegeben.

Darauf hin wird sofort eine neue Zahl dem \gls{dac} übergeben um eine vorher eingestellt Frequenz darzustellen.