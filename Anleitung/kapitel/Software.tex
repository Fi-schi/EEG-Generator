\chapter{Software}

Die Software des Systems gliedert sich in zwei Hauptkomponenten: das Webinterface mit zugehörigem Webserver sowie die Verarbeitungseinheit zur Steuerung des \gls{dac}.
In diesem Kapitel wird der Aufbau beider Softwareteile erläutert. Die konkrete Bedienung der Benutzeroberfläche und Abläufe aus Anwendersicht werden im Anschlusskapitel \textit{Bedienung} beschrieben.

\section{Frontend: Webinterface}

Das Frontend ist eine lokal vom Mikrocontroller bereitgestellte Weboberfläche zur Steuerung und Überwachung des Systems. Die grafische Benutzeroberfläche wurde mit HTML, CSS und JavaScript umgesetzt und als sogenannte Single Page Application (SPA) realisiert. Ziel war eine moderne, intuitive und plattformunabhängige Steuerung über das Netzwerk.

\subsection{Struktur}
Die Weboberfläche besteht aus folgenden Komponenten:

\begin{itemize}
\item Einem Kopfbereich mit Logo und Titel
\item Einer Tab-Navigation zur Auswahl verschiedener Funktionsbereiche
\item Inhaltsbereiche für Datei-Uploads, Dateiverwaltung und Statusanzeigen
\item Upload-Felder zur Übertragung von Dateien an den Mikrocontroller
\item Fortschrittsanzeigen für laufende Dateiübertragungen
\item Menüs zur Kanalzuweisung von Dateien
\end{itemize}

\subsection{Funktion}
Das JavaScript im Frontend übernimmt folgende Aufgaben:

\begin{itemize}
\item Initialisierung und Abfrage von Systemstatus (z.,B. Speicherplatz, Dateiliste)
\item Regelmäßige Aktualisierung des Speicherstatus (alle 5 Sekunden)
\item Hochladen von Dateien per Datei-Dialog oder Drag-and-Drop
\item Anzeige des Uploadfortschritts für jede Datei
\item Interaktive Zuordnung von Dateien zu \gls{dac}-Kanälen
\end{itemize}

Die gesamte Kommunikation mit dem ESP32 erfolgt asynchron per HTTP.

\section{Backend: Mikrocontroller-Firmware}

Das Backend ist als Firmware auf einem ESP32-S3 Mikrocontroller umgesetzt. Es wurde in C++ mit PlatformIO entwickelt und ist modular aufgebaut. Die Firmware übernimmt sowohl die Netzwerkschnittstelle (Webserver), die Dateiverwaltung als auch die Steuerung der Hardware.

\subsection{Systemaufbau}
Die Firmware gliedert sich in folgende Hauptbestandteile:

\begin{itemize}
\item \texttt{main.cpp}: Initialisierung von WLAN, Dateisystem (SPIFFS) und Webserver
\item \texttt{Server.cpp}: Bereitstellung der HTTP-Routen für Datei-Uploads und Speicherfunktionen
\item \texttt{Spannungswandlung.cpp}: Verarbeitung und Umwandlung der digitalen Werte für den \gls{dac}
\item \texttt{PinMapping.cpp}: Zuordnung der Funktionen zu den verwendeten GPIO-Pins
\end{itemize}

\subsection{Webserverfunktionen}
Der Webserver basiert auf der Bibliothek \texttt{ESPAsyncWebServer} und bietet:

\begin{itemize}
\item Bereitstellung der HTML-/JS-Dateien aus dem SPIFFS-Dateisystem
\item Empfang und Speicherung hochgeladener Dateien
\item Auslesen des freien Speicherplatzes
\item Steuerung von GPIOs und Ausgabe an den \gls{dac}
\end{itemize}

\subsection{Datenverarbeitung und DAC-Steuerung}
Sobald der Benutzer im Webinterface den Befehl zur Verarbeitung auslöst, wird der Inhalt der ausgewählten Textdateien in strukturierter Reihenfolge verarbeitet. Die Zahlenwerte in den Dateien werden extrahiert und in Kanal-spezifischen Arrays abgelegt. Dabei erfolgt eine Skalierung auf den 12-Bit-Wertebereich (0 bis 4095) des \gls{dac}, wobei negative Werte (im bipolaren Modus) auf 0 bis 2047, positive auf 2048 bis 4095 abgebildet werden.

Die Umwandlung erfolgt bitweise: Jeder Wert wird in ein 8-Bit-Muster übersetzt, das über entsprechende GPIO-Pins an den \gls{dac} übermittelt wird. Eine Taktsynchronisation stellt sicher, dass jeder Wert korrekt übergeben und konvertiert wird. Die Ausgabe erfolgt mit fester Frequenz, um eine gleichmäßige Signalform zu erzeugen.
