\chapter{Einleitung}
\label{einleitung}

Im Rahmen dieses Projektseminars wurde eine Platine entwickelt, die EEG-Signale generieren und ausgeben kann.
Ziel des Projekts ist es, ein System bereitzustellen, das als anschauliches Beispiel für den Einsatz im Unterricht dient.
Die Platine wird von einem Mikrocontroller gesteuert, der die Signale erzeugt und über einen Digital-Analog-Wandler (DAC) ausgibt.
Zur Ausgabe stehen vier EEG-Kanäle zur Verfügung, denen die Signale gezielt zugewiesen werden können.
Ein weiterer Bestandteil des Projekts ist die Entwicklung eines Webinterfaces, das über die WLAN-Schnittstelle des Mikrocontrollers erreichbar ist.
Die Benutzeroberfläche ermöglicht es, die Signalparameter komfortabel zu konfigurieren und die Verteilung auf die einzelnen Kanäle zu steuern.
Zur Visualisierung der generierten Signale wird ein Oszilloskop verwendet, das die analoge Ausgabe in Echtzeit darstellt.