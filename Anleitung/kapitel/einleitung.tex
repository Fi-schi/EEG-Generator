\chapter{Einleitung}
\label{einleitung}

Im Rahmen dieses Projektseminars wurde eine elektronische Schaltung (Platine) entwickelt, die synthetisch generierte EEG-Signale über vier analoge Ausgänge bereitstellen kann.
Ziel ist die Bereitstellung eines anschaulichen Demonstrationssystems für den Einsatz im naturwissenschaftlich-technischen Unterricht.
Die Signalverarbeitung erfolgt über einen Mikrocontroller, der die digitalen Muster erzeugt und diese über einen Digital-Analog-Wandler (DAC) in analoge Signale umwandelt.
Zur Ausgabe stehen vier EEG-Kanäle zur Verfügung, denen die Signale gezielt zugewiesen werden können.
Ein weiterer Bestandteil des Projekts ist die Entwicklung eines Webinterfaces, das über die WLAN-Schnittstelle des Mikrocontrollers erreichbar ist.
Über die Benutzeroberfläche lassen sich die Signalparameter gezielt einstellen sowie die Kanalzuweisung der Signale verwalten.
Zur Visualisierung der generierten Signale wird ein Oszilloskop verwendet, das die analoge Ausgabe in Echtzeit darstellt.