\documentclass{fiwthesis}

% ========
%  Pakete
% ========

\usepackage{textgreek}           % griechische Buchstaben außerhalb des Math-Mode
\usepackage{amsmath}             % zentrierte Formeln
\usepackage{amssymb}             % erweiterter Formelsatz mathem. Symbole

\usepackage{boldline}            % breitere Linien in Tabellen
\usepackage{booktabs}            % typographisch richtige Tabellen setzen
\usepackage{tabularx}            % Erweiterte Tabellendarstellung
\usepackage{multirow}            % Spalte über mehrere Zeilen oder Spalten ausdehnen
\usepackage{xltabular}           % Zeilenumbrüche in tabularx erlauben

\usepackage{graphicx}            % ermöglicht das Einbinden von Grafiken
\usepackage{subcaption}          % mehrere Bilder in einem Bild
\usepackage{pgfplots}            % Grafiken erzeugen
\usepackage{smartdiagram}        % schnelle und einfache Grafiken
\usepackage{comment}
\usetikzlibrary{positioning}     % bessere Ortsbezeichnung
\usetikzlibrary{shapes}          % typische Formen wie Rechtecke, Ellipsen usw. einfach zeichnen
\usetikzlibrary{intersections}   % Schnittpunkt von Geraden adressieren
\usetikzlibrary{angles, quotes}  % einfacheres Zeichnen von Winkeln
\usetikzlibrary{                 % Symbole für Schaltpläne
  circuits.logic.US,
  circuits.logic.IEC,
  circuits.logic.CDH,
  circuits.ee.IEC
}

\usepackage{lipsum}

% ===========
%  Metadaten
% ===========

\thesis{Projektseminar}
\title{Entwicklung eines EEG Signalgenerators}
\author{Johannes}
\matrnr{123456}
\bdate{08.11.1848}
\bcity{Wismar}
\supervisor{Prof.~Dr.~Simanski}
%\secsupervisor{}
\keywords{Logik, EEG}

% Metadaten in die PDF-Datei schreiben
\makepdfmetadata

% ===============
%  Präambel
% ===============

% PGF Kompatibilitätseinstellung
\pgfplotsset{width=0.95\textwidth,compat=newest}

% % Bibliographie einbinden
\bibliography{quellen}

% Glossar einbinden
\newglossaryentry{nosql}{%
  name = {NoSQL},
  description = {Kurzform für ,,Not Only SQL``; Überbegriff für Datenbanken, die das Konzept relationaler Datenbanken erweitern}
}

\newdualentry{dac}% label
{DAC}% short form
{Digital-Analog-Converter}% long form
{Ein Digital-Analog-Converter (deutsch: Digital-Analog-Wandler) ist ein Chip, der aus einem Digitalem Signal eine Analoge Spannung erzeugt.}% description

% Abkürzungen einbinden
\input{verzeichnisse/abkuerzungen}

% Symbole einbinden
\input{verzeichnisse/symbole}

% Glossar- und Abkürzungsverzeichniserstellung
\makeglossaries{}

% Index erzeugen
\makeindex[
  intoc=true,
  title=Index,
  columns=2]{}
\indexsetup{headers={\indexname}{\indexname}}

% ===============
%  Eigene Makros
% ===============

\newcommand*{\code}[1]{\texttt{#1}}

% ===============
%  Beginn Thesis
% ===============

\begin{document}

% ============
%  "Vorspann"
% ============

% Titelseite
\maketitle

% Aufgabenstellung
% \maketask{\input{kapitel/aufgabenstellung}}

% Abstract
\makeabstract{

Im Rahmen des Projektseminars wurde eine Demonstrationsplatine zur Erzeugung synthetischer EEG-Signale entwickelt. Ziel war es, ein kompaktes und intuitiv bedienbares System zu realisieren, das insbesondere im naturwissenschaftlich-technischen Unterricht zum Einsatz kommen kann. Die Signalverarbeitung übernimmt ein ESP32-S3 Mikrocontroller, der über eine parallele Schnittstelle einen hochauflösenden DAC (DAC8412FPZ) ansteuert. Die vier analogen Ausgangskanäle ermöglichen die gleichzeitige Ausgabe mehrerer Signalverläufe.
Die Konfiguration der Signale erfolgt benutzerfreundlich über eine integrierte Weboberfläche, die ohne zusätzliche Softwareinstallation über WLAN erreichbar ist. Der Aufbau der Schaltung wurde so gewählt, dass die Signale durch Tiefpassfilter geglättet und über Spannungsteiler optional auf den Mikrovoltbereich skaliert werden können. Während der Erprobung zeigte sich, dass sich die Ausgabequalität durch den Einsatz zusätzlicher Filter- oder Versorgungskomponenten weiter optimieren lässt.
Die entwickelte Platine bietet eine solide Basis für didaktische Zwecke sowie eine flexible Plattform für zukünftige Erweiterungen, etwa zur Visualisierung der DAC-Ausgabe im Browser oder zur Einbindung von Signalbibliotheken.

}{
In the context of a project seminar, a demonstration printed circuit board (PCB) was developed for the purpose of generating synthetic electroencephalograms (EEGs). The objective of the project was to design a compact and user-friendly system suitable for use in science and engineering education. The generation of signals is facilitated by an ESP32-S3 microcontroller, which governs a high-resolution DAC (DAC8412FPZ) through a parallel interface. It is evident that the four analog output channels facilitate the simultaneous playback of multiple signal patterns.
The configuration of the signal is managed through a built-in web interface that is accessible via Wi-Fi and does not require the installation of additional software. The circuit design incorporates low-pass filters for signal smoothing and optional voltage dividers for scaling to microvolt levels. As the testing process unfolded, it became evident that the quality of output could be further enhanced through the implementation of advanced filtration techniques and optimisation of the power supply.
The developed system provides a robust foundation for educational applications and a flexible platform for future expansions, such as browser-based signal visualization or the integration of predefined signal libraries.
}

% Inhaltsverzeichnis (Schalter `compact' sorgt für einfachen Zeilenabstand)
\maketoc[compact]

% ==========
%  Textteil
% ==========

% Einleitung
\input{kapitel/Einleitung}

% weitere Kapitel hier jeweils einzeln einbinden
\section{Zielsetzung und Anforderungen}

Ziel des Projekts ist die Entwicklung einer Platine, die EEG-Signale generieren und über vier Kanäle analog ausgeben kann. Dabei soll ein benutzerfreundliches System entstehen, das sich insbesondere für den Einsatz im Unterricht eignet und dort die Funktionsweise von EEG-Signalen veranschaulichen kann.

Die Signalparameter sollen über eine Weboberfläche konfigurierbar sein, ohne dass eine zusätzliche Softwareinstallation auf dem Computer erforderlich ist. Die Oberfläche wird über die integrierte WLAN-Schnittstelle des Mikrocontrollers bereitgestellt.

Hardwareseitig übernimmt ein Mikrocontroller die Generierung der Signale, die anschließend über einen Digital-Analog-Wandler (DAC) ausgegeben werden. Der DAC soll eine hohe Auflösung und Genauigkeit bieten, um eine realistische Darstellung der EEG-Signale zu ermöglichen.

Zur Nachbearbeitung der DAC-Ausgabe wird ein Operationsverstärker als aktiver Tiefpassfilter mit einer Verstärkung von 1 verwendet. Dieser dient der Glättung des Signals und der Reduktion von Rauschen. Da EEG-Signale typischerweise im Mikrovolt-Bereich liegen, wird das Ausgangssignal im Millivolt-Bereich erzeugt und anschließend mittels eines Spannungsteilers im Verhältnis 1:1000 auf den µV-Bereich herunterskaliert.

Die Stromversorgung der Platine erfolgt über eine USB-Schnittstelle. Die Generierung und Ausgabe der Signale soll dabei in Echtzeit erfolgen.

\chapter{Platine}

\section{Komponentenübersichet}

Die wichtigsten Komponenten der Platine sind:

\begin{itemize} 
    \item \textbf{Mikrocontroller:} ESP32-S3 16R8 – zur Signalverarbeitung und Bereitstellung der Weboberfläche. 
    \item \textbf{Digital-Analog-Wandler (DAC):} DAC8412FPZ – zur präzisen Ausgabe der generierten EEG-Signale auf vier Kanälen. 
    \item \textbf{Operationsverstärker:} TL071CDR – dient als aktiver Tiefpassfilter zur Glättung der Ausgangssignale. 
    \item \textbf{Spannungsversorgung:}
    \begin{itemize}
        \item NCV1117DT50RKG – LDO-Spannungsregler für 5 V
        \item TLV75733PDRVR – LDO für 3.3 V Betriebsspannung
        \item LT3580EDD – invertierter DCDC Wandler zur Erzeugung der negativen Spannung
        \item TS4061AILT-1.25 – eine stabile 1{,}25\,V-Referenzspannung 
    \end{itemize}
\end{itemize}

Als Mikrocontroller kommt der ESP32-S3 16R8 zum Einsatz. Dieser bietet mit seinen zwei Kernen die Möglichkeit, die Weboberfläche auf einem Kern und die Signalverarbeitung auf dem anderen Kern zu betreiben. Der integrierte Flash-Speicher von 16MB bietet ausreichend Platz für Firmware und Webinterface.

Zur analogen Ausgabe der Signale wird der \gls{dac} DAC8412FPZ von Analog Devices verwendet.\\
Dieser 12-Bit-DAC zeichnet sich durch hohe Genauigkeit und schnelle Signalverarbeitung aus.
Er verfügt über vier voneinander unabhängige Ausgänge und wird über eine parallele Schnittstelle angesteuert, wodurch eine hohe Datenrate und eine quasi-echtzeitfähige Signalübertragung gewährleistet sind.

Die analoge Nachbearbeitung der Signale erfolgt durch einen Operationsverstärker mit integriertem aktiven Tiefpassfilter. Dieser glättet die DAC-Ausgabe und reduziert hochfrequentes Rauschen.

Zur Erzeugung der notwendigen Versorgungsspannungen kommen ein LDO für 5V, ein 3.3V LDO, sowie ein invertierter DCDC Wandler zum Einsatz. Diese sind notwendig, um eine symmetrische Versorgung von \pm 5V für den DAC Operationsverstärker bereitzustellen.

\section{Schaltungsdesign}
\subsection{Gesamtüberblick}
\begin{figure}[H]
    \centering
    \includegraphics[width=0.8\textwidth]{bilder/Platine_gesamt.png}
    \caption{Gesamtübersicht der Platine}
    \label{fig:gesamtuebersicht}
\end{figure}
Die Platine ist in mehrere Funktionsblöcke unterteilt, die jeweils für spezifische Aufgaben zuständig sind. Diese Blöcke sind:
\begin{itemize}
    \item \textbf{Mikrocontroller-Block:} Enthält den ESP32-S3 16R8, der die Signalverarbeitung und die Weboberfläche steuert.
    \item \textbf{DAC-Block:} Beinhaltet den DAC8412FPZ, der die digitalen Signale in analoge Signale umwandelt.
    \item \textbf{Operationsverstärker-Block:} Besteht aus dem TL071CDR, der als aktiver Tiefpassfilter fungiert.
    \item \textbf{Spannungsversorgungsblock:} Umfasst die LDOs, den DCDC Wandler und die beiden Shunt Spannungsreferenzen zur Bereitstellung der benötigten Spannungen und Referenzen.
    \item \textbf{USB-Block:} Dient der Stromversorgung der Platine über eine USB-Schnittstelle.
\end{itemize}

\subsection{Mikrocontroller-Block}
\begin{figure}[H]
    \centering
    \includegraphics[width=0.8\textwidth]{bilder/Mikrocontroller_Block.png}
    \caption{Mikrocontroller-Block der Platine}
    \label{fig:mikrocontroller_block}
\end{figure}

Der Mikrocontroller-Block enthält den ESP32-S3 16R8, der die zentrale Steuereinheit der Platine darstellt. Er ist für die Signalverarbeitung, die Bereitstellung der Weboberfläche und Speicherung der EEG-Simulationsdaten verantwortlich. Der Mikrocontroller kommuniziert über eine parallele Schnittstelle mit dem \gls{dac} und steuert dessen Ausgänge.\\
Die Programmierung des Mikrocontrollers erfolgt in C++ unter Verwendung der Arduino-IDE mit PlatformIO. 

\subsection{DAC-Block}
\begin{figure}[H]
    \centering
    \includegraphics[width=0.8\textwidth]{bilder/DAC.png}
    \caption{DAC-Block der Platine}
    \label{fig:dac_block}
\end{figure}

Der DAC-Block enthält den DAC8412FPZ mit seiner speziellen Beschaltung.
Der \gls{dac} wandelt die digitalen Signale des Mikrocontrollers in analoge Signale umwandelt. Hierbei werden die Signal mit einer Parallelen-Schnittstelle übertragen, die es durch gleichzeitiges bereitstellen der einzelnen Digitalen Bits, ermöglichst eine hohe Datenrate bereitzustellen. Der DAC8412FPZ bietet eine Auflösung von 12 Bit und vier unabhängige Ausgänge, die jeweils für die Ausgabe eines EEG-Signals verwendet werden können.\\
Die Signale werden vom Mikrocontroller bereitgestellt und über die parallele Schnittstelle an den DAC übertragen. Der DAC wandelt diese Signale in analoge Spannungen um, die dann weiterverarbeitet werden.\\

\subsection{Operationsverstärker-Block}
\begin{figure}[H]
    \centering
    \includegraphics[width=0.8\textwidth]{bilder/Operationsverstaerker_Block.png}
    \caption{Operationsverstärker-Block der Platine}
    \label{fig:operationsverstaerker_block}
\end{figure}

Der Operationsverstärker-Block enthält den TL071CDR, der als aktiver Tiefpassfilter fungiert. Dieser glättet die analogen Signale des \gls{dac} und reduziert hochfrequentes Rauschen. Der Operationsverstärker ist so konfiguriert, dass er eine Verstärkung von 1 bietet, um die Ausgangsspannungen im Millivolt-Bereich zu halten.\\
Im weiter inst noch möglich einen weiterer Operationsverstärker zu verbauen der als Spannungsfolger fungieren kann, um die Ausgangsimpedanz zu reduzieren und die Signalqualität zu verbessern. Standardmäßig ist dieser nicht verbaut und mit einem Jumper überbrückt. 

\subsection{Spannungsversorgung}
\begin{figure}[H]
    \centering
    \includegraphics[width=0.8\textwidth]{bilder/Bipolar_Power.png}
    \caption{Spannungsversorgungsblock der Platine}
    \label{fig:spannungsversorgung}
\end{figure}
Die Spannungsversorgung der Platine erfolgt über eine USB-Schnittstelle, die 5V bereitstellt. 
Diese wird durch den LDO NCV1117DT50RKG auf 5V für den DAC und den Operationsverstärker geglättet.\\
Für die Versorgung des Mikrocontrollers ist der TLV75733PDRVR zuständig, der eine stabile 3.3V Versorgungsspannung liefert.
Da der \gls{dac} DAC8412FPZ eine symmetrische Versorgungsspannung von \pm 5V benötigt, wird ein invertierter DCDC Wandler LT3580EDD verwendet, der aus der 5V Versorgungsspannung eine negative Spannung von -5V erzeugt. Die wird zusätzlich auch für die Versorgung der Operationsverstärker genutzt.\\
Da der \gls{dac} eine Referenzspannung benötigt, wird eine stabile \pm 1.25V Referenzspannung durch zwei TS4061AILT-1.25 bereitgestellt. 

\subsection{USB-Block}
\begin{figure}[H]
    \centering
    \includegraphics[width=0.8\textwidth]{bilder/USBC_Port.png}
    \caption{USB-Block der Platine}
    \label{fig:usb_block}
\end{figure}
Der USB-Block dient der Stromversorgung der Platine über eine USB-Schnittstelle. Er enthält die notwendigen Schaltungen, um die 5V Versorgungsspannung der USB-Schnittstelle einzustellen und die Kommunikation mit dem Mikrocontroller zu ermöglichen.\\
\chapter{Software}

Die Software des Systems gliedert sich in zwei Hauptkomponenten: das Webinterface mit zugehörigem Webserver sowie die Verarbeitungseinheit zur Steuerung des \gls{dac}.
In diesem Kapitel wird der Aufbau beider Softwareteile erläutert. Die konkrete Bedienung der Benutzeroberfläche und Abläufe aus Anwendersicht werden im Anschlusskapitel \textit{Bedienung} beschrieben.

\section{Frontend: Webinterface}

Das Frontend ist eine lokal vom Mikrocontroller bereitgestellte Weboberfläche zur Steuerung und Überwachung des Systems. Die grafische Benutzeroberfläche wurde mit HTML, CSS und JavaScript umgesetzt und als sogenannte Single Page Application (SPA) realisiert. Ziel war eine moderne, intuitive und plattformunabhängige Steuerung über das Netzwerk.

\subsection{Struktur}
Die Weboberfläche besteht aus folgenden Komponenten:

\begin{itemize}
\item Einem Kopfbereich mit Logo und Titel
\item Einer Tab-Navigation zur Auswahl verschiedener Funktionsbereiche
\item Inhaltsbereiche für Datei-Uploads, Dateiverwaltung und Statusanzeigen
\item Upload-Felder zur Übertragung von Dateien an den Mikrocontroller
\item Fortschrittsanzeigen für laufende Dateiübertragungen
\item Menüs zur Kanalzuweisung von Dateien
\end{itemize}

\subsection{Funktion}
Das JavaScript im Frontend übernimmt folgende Aufgaben:

\begin{itemize}
\item Initialisierung und Abfrage von Systemstatus (z.,B. Speicherplatz, Dateiliste)
\item Regelmäßige Aktualisierung des Speicherstatus (alle 5 Sekunden)
\item Hochladen von Dateien per Datei-Dialog oder Drag-and-Drop
\item Anzeige des Uploadfortschritts für jede Datei
\item Interaktive Zuordnung von Dateien zu \gls{dac}-Kanälen
\end{itemize}

Die gesamte Kommunikation mit dem ESP32 erfolgt asynchron per HTTP.

\section{Backend: Mikrocontroller-Firmware}

Das Backend ist als Firmware auf einem ESP32-S3 Mikrocontroller umgesetzt. Es wurde in C++ mit PlatformIO entwickelt und ist modular aufgebaut. Die Firmware übernimmt sowohl die Netzwerkschnittstelle (Webserver), die Dateiverwaltung als auch die Steuerung der Hardware.

\subsection{Systemaufbau}
Die Firmware gliedert sich in folgende Hauptbestandteile:

\begin{itemize}
\item \texttt{main.cpp}: Initialisierung von WLAN, Dateisystem (SPIFFS) und Webserver
\item \texttt{Server.cpp}: Bereitstellung der HTTP-Routen für Datei-Uploads und Speicherfunktionen
\item \texttt{Spannungswandlung.cpp}: Verarbeitung und Umwandlung der digitalen Werte für den \gls{dac}
\item \texttt{PinMapping.cpp}: Zuordnung der Funktionen zu den verwendeten GPIO-Pins
\end{itemize}

\subsection{Webserverfunktionen}
Der Webserver basiert auf der Bibliothek \texttt{ESPAsyncWebServer} und bietet:

\begin{itemize}
\item Bereitstellung der HTML-/JS-Dateien aus dem SPIFFS-Dateisystem
\item Empfang und Speicherung hochgeladener Dateien
\item Auslesen des freien Speicherplatzes
\item Steuerung von GPIOs und Ausgabe an den \gls{dac}
\end{itemize}

\subsection{Datenverarbeitung und DAC-Steuerung}
Sobald der Benutzer im Webinterface den Befehl zur Verarbeitung auslöst, wird der Inhalt der ausgewählten Textdateien in strukturierter Reihenfolge verarbeitet. Die Zahlenwerte in den Dateien werden extrahiert und in Kanal-spezifischen Arrays abgelegt. Dabei erfolgt eine Skalierung auf den 12-Bit-Wertebereich (0 bis 4095) des \gls{dac}, wobei negative Werte (im bipolaren Modus) auf 0 bis 2047, positive auf 2048 bis 4095 abgebildet werden.

Die Umwandlung erfolgt bitweise: Jeder Wert wird in ein 8-Bit-Muster übersetzt, das über entsprechende GPIO-Pins an den \gls{dac} übermittelt wird. Eine Taktsynchronisation stellt sicher, dass jeder Wert korrekt übergeben und konvertiert wird. Die Ausgabe erfolgt mit fester Frequenz, um eine gleichmäßige Signalform zu erzeugen.


\chapter{Bedienung}
\label{chap:bedienung}

Die Bedienung der EEG-Demonstrationsplatine erfolgt vollständig über eine browserbasierte Weboberfläche, die vom Mikrocontroller bereitgestellt wird. Es ist keine zusätzliche Softwareinstallation notwendig – ein aktueller Webbrowser genügt. Im Folgenden wird der typische Bedienablauf Schritt für Schritt beschrieben.

\section{Verbindung zur Platine}

Nach dem Einschalten der Platine erzeugt der ESP32-S3 einen eigenen WLAN-Hotspot mit dem Netzwerknamen \textit{EEG-Simulator} und dem Netzwerkpasswort \textit{123456789}.
Nutzerinnen und Nutzer verbinden sich mit diesem WLAN, das keine Internetverbindung benötigt. Anschließend kann die Weboberfläche über die IP-Adresse \texttt{192.168.4.1} im Browser geöffnet werden.



\section{Weboberfläche}

Die Weboberfläche ist in mehrere Bereiche unterteilt:

\begin{itemize}
  \item \textbf{Dateiupload:} Ermöglicht das Hochladen von .txt-Dateien, die Zahlenwerte zur DAC-Ausgabe enthalten. Der Upload erfolgt per Drag-and-Drop oder über einen Dateiauswahldialog.
  \item \textbf{Dateiverwaltung:} Zeigt eine Liste aller hochgeladenen Dateien an. Die Reihenfolge kann per Drag-and-Drop verändert werden und bestimmt die spätere Abspielreihenfolge.
  \item \textbf{Kanalauswahl:} Jeder Datei kann ein DAC-Ausgangskanal zugewiesen werden (A bis D). Die Zuordnung erfolgt über ein Auswahlmenü neben dem Dateinamen.
  \item \textbf{Signalverarbeitung:} Durch Auswahl und anschließenden Klick auf den Button \textit{„Verarbeitung anstoßen“} wird der Inhalt der gewählten Dateien eingelesen, auf korrekte Struktur überprüft und für die DAC-Ausgabe vorbereitet.
  \item \textbf{Signalwiedergabe:} Der Button \textit{„Abspielen“} startet die Ausgabe der Signale über die vier DAC-Kanäle. Die Daten werden dabei taktgenau an den DAC übertragen und in analoge Spannungen gewandelt.
\end{itemize}

\section{Ablauf der Signalerzeugung}

Nach dem Hochladen und der Auswahl der relevanten Dateien läuft die Erzeugung der EEG-Signale wie folgt ab:

\begin{enumerate}
  \item Die enthaltenen Zahlenwerte werden aus den Textdateien extrahiert.
  \item Negative und positive Werte werden auf den 12-Bit-Wertebereich des DACs abgebildet (0–2047 für negativ, 2048–4095 für positiv).
  \item Die konvertierten Werte werden kanalweise in interne Speicherarrays geladen.
  \item Die Ausgabefrequenz bestimmt das zeitliche Intervall zwischen den DAC-Werten.
  \item Bei Klick auf \textit{„Abspielen“} beginnt die kontinuierliche, synchrone Ausgabe.
\end{enumerate}

\section{Hinweise zur Verwendung}

\begin{itemize}
  \item Nach Änderungen an der Dateireihenfolge oder Kanalauswahl muss die Verarbeitung erneut angestoßen werden.
  \item Eine stabile Stromversorgung über USB-C ist für die gleichmäßige Ausgabe unerlässlich.
  \item Die Signale können mit einem Oszilloskop am Ausgang überwacht werden.
\end{itemize}

% Schluss
\chapter{Zusammenfassung und Ausblick}
\label{schluss}

\section{Zusammenfassung}

Im Rahmen dieses Projekts wurde eine funktionsfähige Demonstrationsplatine zur Erzeugung synthetischer EEG-Signale entwickelt. Ziel war es, ein intuitiv bedienbares und technisch robustes Gerät für den Einsatz im Unterricht bereitzustellen. 

Die entwickelte Schaltung basiert auf einem ESP32-S3 Mikrocontroller in Kombination mit einem hochauflösenden DAC (DAC8412FPZ). Über vier unabhängige Ausgänge können benutzerdefinierte Signale analog ausgegeben und über Operationsverstärker geglättet werden. Die Parametrierung erfolgt über eine browserbasierte Weboberfläche, die vollständig ohne zusätzliche Softwareinstallation auskommt.

Während der Testphase zeigten sich einige Einschränkungen: Die vom invertierenden DC/DC-Wandler erzeugte negative Versorgungsspannung enthielt ein störendes hochfrequentes Signal. Dieses lässt sich jedoch durch einen nachgeschalteten Low-Dropout-Regler (LDO) wirksam unterdrücken. Alternativ wurde erfolgreich eine Versorgung über zwei symmetrisch verschaltete Lithium-Akkus getestet, wodurch die Störungen vollständig vermieden wurden.

Die Signalqualität im Mikrovoltbereich war zunächst stark von Rauschanteilen überlagert. Mittels spektraler Analyse (FFT) konnte das Nutzsignal jedoch nachgewiesen werden. Durch das Überbrücken des Spannungsteilers mittels 0-Ohm-Jumpern lässt sich das Signal alternativ auch im Millivoltbereich nutzen, was eine deutlich einfachere Auswertung per Oszilloskop ermöglicht.

\section{Ausblick}

Die derzeitige Version erfüllt alle grundlegenden Anforderungen an eine signalgebende EEG-Demoplatine. Dennoch ergeben sich mehrere sinnvolle Erweiterungs- und Optimierungsmöglichkeiten:

\begin{itemize}
  \item \textbf{Verbesserung der Filterung:} Der Einsatz eines höherwertigen analogen Tiefpassfilters (z.\,B. zweiter Ordnung oder Sallen-Key-Topologie) könnte die Rauschunterdrückung weiter verbessern. Dies muss jedoch experimentell verifiziert werden.
  \item \textbf{Stromversorgung:} Eine saubere Glättung der bipolare Versorgung über LDOs oder Versorgung mittels Akkus wäre langfristig stabiler als ein DC/DC-Wandler.
  \item \textbf{Signalbibliothek:} Eine optionale, nicht im Projekt enthaltene Sammlung vordefinierter Signalformen (z.,B. Alpha- oder Betawellen) könnte zukünftig über die Weboberfläche eingebunden werden.
  \item \textbf{Live-Visualisierung:} Eine grafische Darstellung der aktuellen DAC-Ausgabe direkt auf der Weboberfläche würde die Bedienung deutlich intuitiver gestalten. Dadurch könnten Nutzerinnen und Nutzer die Signalform in Echtzeit mitverfolgen, ohne auf externe Messgeräte wie ein Oszilloskop angewiesen zu sein.
\end{itemize}

Insgesamt bietet die entwickelte Platine eine solide Grundlage für den universitären Einsatz und stellt eine flexible Plattform für weitere didaktische und technische Weiterentwicklungen dar.


% =========
%  Anlagen
% =========

%\begin{appendices}

  %\input{anlagen/beispiel}

%\end{appendices}

% ===============
%  Verzeichnisse
% ===============

% Verzeichnisse mit einzeiligem Zeilenabstand
\singlespacing

% Literaturverzeichnis
\listofreferences

% Abbildungsverzeichnis einfügen
\listoffigures
% Abkürzungsverzeichnis
%\listofacronyms

% Symbolverzeichnis
%\listofsymbols

% falls ein anderer Glossar-Stil genutzt wird und die zweite Spalte zu schmal ist:
% \setlength{\glsdescwidth}{0.8\linewidth}

% Glossar einfügen
%\printglossary

% Index einfügen
\printindex

% wieder auf 1½-fachen Zeilenabstand umschalten
\normalspacing

% =========================================
%  Selbstständigkeitserklärung, CD, Thesen
% =========================================

% Selbstständigkeitserklärung
\makedeclarationofindependence


\end{document}

% =============
%  Ende Thesis
% =============
